import { useRoutes} from "react-router-dom";
import Home from "./component/Home";
import About from "./component/About";
import NoPage from "./component/NoPage";
import LayoutHome from "./component/LayoutHome";
import Contact from "./component/Contact";

  //   <>
  //     <Routes>
  //       <Route index  element={<Home/>} />
  //       <Route path="/about/:id" element={<div><h2>About5</h2> <About/></div>} />
  //     </Routes>
  //   </>



//news
//   const routers = useRoutes([
//     {
//       path: '*',
//       element: <NotFound />,
//     },
//     {
//       path: '/',
//       element: <MainPage />,
//     },
//     {
//       path: '/ogs-new',
//       element: <LayoutPage />,
//       children: [
//         {
//           index: true,
//           element: <HomePage />,
//         },
//         {
//           path: 'news/:id',
//           element: <AllNew />,
//         },
//         {
//           path: 'detail-news/:id',
//           element: <DetailNew />,
//         },
//       ],
//     },
//   ]);
//   return routers
// }

const [all, setAll] = useState([1, 2, 3, 4, 5]);
useEffect(() => {
  const deleteData = () => {
    if (all.length > 0) {
      // ลบข้อมูลตัวแรกใน array
      const newData = [...all]; // สร้างคัดลอกข้อมูลเพื่อป้องกันการแก้ไขข้อมูลตรงๆ
      newData.shift(); // ลบข้อมูลตัวแรกใน array ใหม่

      // อัพเดทข้อมูลใหม่ใน state
      setAll(newData);

      // เรียกฟังก์ชัน deleteData อีกครั้งหลังจากผ่านไปเวลาที่กำหนด
      setTimeout(deleteData, 50000); // เวลาใน milliseconds (ในที่นี้คือ 1 วินาที)
    } else {
      console.log("All data deleted");
    }
  };

  // เริ่มการลบข้อมูลเมื่อคอมโพเนนต์โหลด
  deleteData();
}, [all]); // ให้ useEffect เรียกใช้งานเมื่อ all เปลี่ยนแปลง